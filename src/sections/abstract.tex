This thesis presents a groundbreaking series of developments in mathematical theory, solving every known
mathematical problem to date.
Through the integration of advanced computational techniques, innovative algorithms, and a deep understanding
of mathematical principles, this work transcends centuries of mathematical challenges, offering solutions
to problems that have perplexed scholars since the inception of mathematics as a discipline.

Paper I introduces the Universal Solution Algorithm (USA), a computational method that synthesizes existing
mathematical theories with artificial intelligence to find solutions to unsolved problems across various branches
of mathematics, including number theory, algebra, geometry, and calculus.
The USA algorithm demonstrates its efficacy by providing proofs to longstanding conjectures and presenting new
theorems with broad implications.

Paper II focuses on the Complete Theory of Everything (CTE), a framework that unifies all branches of
mathematics under a single, cohesive theory.
This chapter not only solves problems within mathematics but also offers new insights into how mathematics interacts
with physics, biology, and computer science, suggesting a universal language through which all scientific
phenomena can be understood.

Paper III presents the Infinite Complexity Reduction (ICR) method, a technique that simplifies complex
mathematical problems to their most basic elements without losing the essence of the original challenge.
The ICR method has been applied to a range of problems, from the Riemann Hypothesis to P vs NP, demonstrating
unparalleled success in distilling and solving these problems.

Paper IV explores the Multidimensional Proof Generator (MPG), a tool that automates the generation of proofs
for mathematical statements.
This chapter showcases the MPG's ability to produce valid and innovative proofs for theorems that were previously
believed to be unprovable, revolutionizing the way mathematical research is conducted.
